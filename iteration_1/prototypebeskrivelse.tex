Link til prototype: \url{https://xd.adobe.com/view/784ff988-1379-4ba1-7d13-9756c215e054-0e7e/}

Vores prototype har fokus på 3 vigtige områder, som vi mener er gode til og motivere patienten. \\

\begin{itemize}
	\item Login skærmen
	\begin{itemize}
		\item Login skærmen vil tillade at man logger ind med den nøgle som man er tilknyttet.
		\item Når man er logget ind vil man så blive ledt til enten dyret eller grafen, hvilket man vil kunne vælge første gang man logger ind, eller i indstillingerne.
	\end{itemize}
	\item Dyret
	\begin{itemize}
		\item Dyret vil fungere som en slags kæledyr. Hver dag vil den skulle holdes i live, og dette opnås ved at opnå ens daglige mål. Jo flere gange man misser sine daglige mål, desto værre får ens dyr det.
		\item Dyret vil påminde en om hvor godt man klarer sig, ved små beskeder til patienten, skrevet fra dyrets synspunkt.
	\end{itemize}
	\item Trofæer
	\begin{itemize}
		\item For at hjælpe til og være motiveret, vil vi have belønninger til folk der opfylder deres mål. Det kan for eksempel være at man får et trofæ for at opfylde ens mål hver dag i en måned.
		\item Disse trofæer vil så give point der måske vil kunne bruges til at købe sjove ting for til ens dyr, eller andet virtuelt.
		\item Trofæer vil så kunne deles på sociale medier, for at vise andre, måske familie, hvor godt man klarer sig.
	\end{itemize}
	\item Grafoversigt
	\begin{itemize}
		\item Til de folk der gerne vil have en klinisk oversigt over deres ydeevne, vil det være muligt at kunne se grafer over ens aktivitet.
		\item Til patienter der kun ønsker denne, vil ens trofæer stadig være mulige, og måske også muligt at købe farver til ens grafer og andet.
	\end{itemize}
\end{itemize}

På både trofæerne, grafoversigt og dyret, vil der være en menu der giver en yderligere adgang til indstillinger, informationer og handikap funktioner.
